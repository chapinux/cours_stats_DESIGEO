\documentclass{beamer}
\usepackage{mathrsfs}  
\usepackage{xcolor}
\usepackage{setspace}
\usepackage{comment}

% config du thgeme metropolis
\usetheme[progressbar=frametitle,block=fill, titleformat=smallcaps,sectionpage=progressbar,]{metropolis}



\title{Projets Probabilités, Statistiques, Analyse Spatiale}
\subtitle{}
\date{2021-2022}
\author{Paul Chapron \textsuperscript{1} \& Yann Ménéroux \textsuperscript{1} \& Juste Rimbault \textsuperscript{1}}
\institute{ \textsuperscript{1}IGN-ENSG-UGE}



%definition de la couleur du texte dans la balise \alert{}
\definecolor{vertIGN}{HTML}{96C31E} % vert IGN %vrai valeur #97BE0D
\setbeamercolor{alerted text}{fg=vertIGN}

\definecolor{grisIGN}{HTML}{22292F} % Gris IGN tiré vers le noir 
\setbeamercolor{background canvas}{bg=grisIGN}




% code pour placer le log ENSG dans le bandeau de titre 
\makeatletter
\setbeamertemplate{frametitle}{%
  \nointerlineskip%
  \begin{beamercolorbox}[%
      wd=\paperwidth,%
      sep=0pt,%
      leftskip=\metropolis@frametitle@padding,%
      rightskip=\metropolis@frametitle@padding,%
    ]{frametitle}%
  \metropolis@frametitlestrut@start%
  \insertframetitle%
  \nolinebreak%
  \metropolis@frametitlestrut@end%
  \hfill
  \raisebox{-0.6ex}{\includegraphics[height=4ex,keepaspectratio]{img/logoENSG_small.jpg}}
  \end{beamercolorbox}%
}
\makeatother




% logo ENSG première page 
\titlegraphic{\vspace{4cm}\flushright\includegraphics[width=2cm,height=2cm]{img/logoENSG_big.png}} 



\begin{document}
\metroset{background=dark} % change background theme according to manual
\maketitle

\begin{frame}{Dans les cours précédents ... }

\begin{itemize}
  \item Statistiques / Géostatistiques 
  \item Probabilités 
  \item Analyse spatiale 
  \item Manipulations avec R 
\end{itemize}

\end{frame}


\begin{frame}{L'idée du projet}

Mobiliser un mélange des compétences et connaissances acquises pour \alert{répondre à une question} sur des données spatiales, en groupe.




\end{frame}






\begin{frame}{Contraintes}
$\rightarrow $ Groupes de 3 étudiant·e·s max \\ 
\vspace{1cm}

$\rightarrow $ Rapport de 10 pages max\\
\vspace{1cm}

$\rightarrow $ Soutenance de 15 minutes, tout le monde parle, 16 décembre après-midi, amphi Picard 

\end{frame}

\begin{frame}{Données idéales}

\begin{itemize}
\item spatialisées (ou spatialisables)
\item quantitatives (au moins deux ou trois! )
\item multiples (nombre de variables distinctes)
\item nombreuses (nombre d'individus de la population)
\item ouvertes 
\end{itemize}
\end{frame}


\begin{frame}{Exemples : Projets passés}
\begin{small}
Comment établir un profil de communes canadiennes à partir du temps de trajet domicile-travail des habitants ? 



Quels sont les facteurs qui influencent le nombre d’infractions recensées par les forces de l’ordre ?

Peut-on expliquer le taux de suicide d'un pays à l'aide de données socio-économiques ? 

Quels sont les facteurs qui font d'un film un film à succès ? 

Estimation et prédiction spatio-temporelle de la consommation électrique à Paris
de 2018 à 2025.


Classements internationaux  des écoles supérieurs : Constitution, évolution et taux d’emploi des diplomés.
\end{small}

\end{frame}



\begin{comment}
\begin{frame}{Animation}
  \begin{itemize}[<+- | alert@+>]
    \item \alert<4>{This is\only<4>{ really} important}
    \item Now this
    \item And now this
  \end{itemize}
\end{frame}



\begin{frame}[standout]
Mono message sur une diapo
\end{frame}
\end{comment}

\end{document}